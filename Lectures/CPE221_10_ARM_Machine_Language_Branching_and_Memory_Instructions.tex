\documentclass[aspectratio=169]{beamer}
\usepackage{UAH-theme}
\usepackage{minted}
\usepackage{xcolor}
% Use a placeholder if you don't have the actual logo
% Create a simple UAH logo placeholder
\usepackage{tikz}
\usepackage{array}
\usepackage{multirow}
\usepackage{array}
\usepackage{colortbl}
\usepackage{booktabs}
\usepackage{graphicx} 
\usepackage[many]{tcolorbox}


\usetikzlibrary{arrows.meta, positioning}
\definecolor{stackaddr}{RGB}{173,216,230}
\definecolor{stackval}{RGB}{255,228,181}
\definecolor{stackdesc}{RGB}{144,238,144}
\definecolor{addr}{RGB}{173,216,230}
\definecolor{val}{RGB}{255,228,181}
\definecolor{desc}{RGB}{144,238,144}

\newcommand{\UAHLogoPlaceholder}{%
\begin{tikzpicture}[scale=0.15]
\draw[fill=UAHred,draw=none] (0,0) circle (5);
\draw[white,line width=0.5mm] (0,0) circle (4);
\draw[white,line width=0.5mm] (-2,2) -- (2,2);
\draw[white,line width=0.5mm] (-2,-2) -- (2,-2);
\draw[white,line width=0.5mm] (0,4) -- (0,-4);
\end{tikzpicture}%
}

\definecolor{frame1}{RGB}{255, 230, 230}
\definecolor{frame2}{RGB}{230, 255, 230}
\definecolor{frame3}{RGB}{230, 230, 255}
\definecolor{frame4}{RGB}{255, 255, 230}

\definecolor{androidBlue}{HTML}{8AB4F8}
\definecolor{androidBlueLight}{HTML}{E8F0FE}
\definecolor{androidGreen}{HTML}{81C995}
\definecolor{androidGreenLight}{HTML}{E6F4EA}

% Red variants - based on Android's "error" and warning colors
\definecolor{androidRed}{HTML}{F28B82}
\definecolor{androidRedLight}{HTML}{FADAD7}

% Yellow variants - based on Android's accent/warning colors
\definecolor{androidYellow}{HTML}{FDD663}
\definecolor{androidYellowLight}{HTML}{FEF7E0}

% Purple variants - based on Android's system UI accents
\definecolor{androidPurple}{HTML}{D7AEFB}
\definecolor{androidPurpleLight}{HTML}{F4EAFC}

% Orange variants - based on Android's notification colors
\definecolor{androidOrange}{HTML}{FCAD70}
\definecolor{androidOrangeLight}{HTML}{FEEADC}

% Teal variants - based on Android's Material You palette
\definecolor{androidTeal}{HTML}{78D9EC}
\definecolor{androidTealLight}{HTML}{E6F6F9}

% Gray variants - based on Android's neutral colors
\definecolor{androidGray}{HTML}{DADCE0}
\definecolor{androidGrayLight}{HTML}{F1F3F4}



\usemintedstyle{default}
\setminted{
fontsize=\footnotesize,
frame=single,
linenos=true,
breaklines=true,
autogobble=true
}


\title{10 ARM Machine Code: Branching and Memory Instructions}
\subtitle{CPE 221}
\author{Rahul Bhadani}
\institute{The University of Alabama in Huntsville}
\date{\today}


\begin{document}

\begin{frame}

    \titlepage

\end{frame}

\begin{frame}{Table of Content}
    \tableofcontents
\end{frame}


\section{Memory Instruction}

\begin{frame}
    \sectionpage
\end{frame}

\begin{frame}{Memory Instructions}
\begin{center}
    \renewcommand{\arraystretch}{1.5}
    % Instruction format table
    \begin{tabular}{|p{3.0em}|p{3.0em}|p{1.5em}|p{1.5em}|p{1.5em}|p{1.5em}|p{1.5em}|p{1.5em}|p{3.0em}|p{3.0em}|p{3.0em}|}
    \hline
    31:28 & 27:26 & 25 & 24 & 23 & 22 & 21 & 20 & 19:16 & 15:12 & 11:0\\
    \hline
    cond & op & \=I & P & U & B & W & L & Rn & Rd & Src2\\
    \hline
    4 bits & 2 bits & 1 bit & 1 bit & 1 bit & 1 bit &  1 bit & 1 bit & 4 bits & 4 bits & 12 bits\\
    \hline
    \end{tabular}

    \vspace{0.5cm}
If \=I = 1 (no immediate) but with register shifting:

    \begin{tabular}{|p{6em}|p{6em}|p{6em}|p{5em}|}
        \hline
        Bits 11:7 & Bits 6:5 & Bit 4 & Bits 3:0\\
        \hline
        shamt5 & sh & 1/0 & Rm \\
        \hline
        5-bit shift amount & specific shift instructions & 1 in the textbook, 0 in the cpulator &  Register used for shifting\\
        \hline
    \end{tabular}
\end{center}
\end{frame}

\begin{frame}{Memory Instructions}
    \begin{center}
        \renewcommand{\arraystretch}{1.5}
        % Instruction format table
        \begin{tabular}{|p{3.0em}|p{3.0em}|p{1.5em}|p{1.5em}|p{1.5em}|p{1.5em}|p{1.5em}|p{1.5em}|p{3.0em}|p{3.0em}|p{3.0em}|}
        \hline
        31:28 & 27:26 & 25 & 24 & 23 & 22 & 21 & 20 & 19:16 & 15:12 & 11:0\\
        \hline
        cond & op & \=I & P & U & B & W & L & Rn & Rd & Src2\\
        \hline
        4 bits & 2 bits & 1 bit & 1 bit & 1 bit & 1 bit &  1 bit & 1 bit & 4 bits & 4 bits & 12 bits\\
        \hline
        \end{tabular}
    
        \vspace{0.5cm}
    If \=I = 1 (no immediate) no shifting, just a register:
    
        \begin{tabular}{|p{6em}|p{6em}|p{6em}|p{5em}|}
            \hline
            Bits 11:7 & Bits 6:5 & Bit 4 & Bits 3:0\\
            \hline
            shamt5 & sh & 0 & Rm \\
            \hline
            00000 & 00 & 0 (default) &  Register \\
            \hline
        \end{tabular}
    \end{center}
    \end{frame}

\begin{frame}{Memory Instructions}
    \begin{center}
        \renewcommand{\arraystretch}{1.5}
        % Instruction format table
        \begin{tabular}{|p{3.0em}|p{3.0em}|p{1.5em}|p{1.5em}|p{1.5em}|p{1.5em}|p{1.5em}|p{1.5em}|p{3.0em}|p{3.0em}|p{3.0em}|}
        \hline
        31:28 & 27:26 & 25 & 24 & 23 & 22 & 21 & 20 & 19:16 & 15:12 & 11:0\\
        \hline
        cond & op & \=I & P & U & B & W & L & Rn & Rd & Src2\\
        \hline
        4 bits & 2 bits & 1 bit & 1 bit & 1 bit & 1 bit &  1 bit & 1 bit & 4 bits & 4 bits & 12 bits\\
        \hline
        \end{tabular}
    
        \vspace{0.5cm}
    If \=I =0 (immediate):
    
        \begin{tabular}{|p{12em}|}
            \hline
            Bits 11:0\\
            \hline
            imm12 \\
            \hline
            12 bit immediate\\
            \hline
        \end{tabular}
    \end{center}
    \end{frame}

\begin{frame}{Memory Instructions}
    \begin{center}
        \renewcommand{\arraystretch}{1.5}
        % Instruction format table
        \begin{tabular}{|p{3.0em}|p{3.0em}|p{1.5em}|p{1.5em}|p{1.5em}|p{1.5em}|p{1.5em}|p{1.5em}|p{3.0em}|p{3.0em}|p{3.0em}|}
        \hline
        31:28 & 27:26 & 25 & 24 & 23 & 22 & 21 & 20 & 19:16 & 15:12 & 11:0\\
        \hline
        cond & op & \=I & P & U & B & W & L & Rn & Rd & Src2\\
        \hline
        4 bits & 2 bits & 1 bit & 1 bit & 1 bit & 1 bit &  1 bit & 1 bit & 4 bits & 4 bits & 12 bits\\
        \hline
        \end{tabular}
    
        \vspace{0.5cm}
    
        I = Immediate, P = Pre-index, U = Add, W = Writeback, L = Load, B = Byte
    \end{center}
    \end{frame}

    \begin{frame}{Memory Instructions}
        \begin{center}
            \renewcommand{\arraystretch}{1.2}
            % Instruction format table
            \begin{tabular}{|p{3.0em}|p{3.0em}|p{1.5em}|p{1.5em}|p{1.5em}|p{1.5em}|p{1.5em}|p{1.5em}|p{3.0em}|p{3.0em}|p{3.0em}|}
            \hline
            31:28 & 27:26 & 25 & 24 & 23 & 22 & 21 & 20 & 19:16 & 15:12 & 11:0\\
            \hline
            cond & op & \=I & P & U & B & W & L & Rn & Rd & Src2\\
            \hline
            4 bits & 2 bits & 1 bit & 1 bit & 1 bit & 1 bit &  1 bit & 1 bit & 4 bits & 4 bits & 12 bits\\
            \hline
            \end{tabular}
            \vspace{0.5cm}
        
            \begin{tabular}{|p{3.0em}|p{3.0em}|p{8.0em}|}
                \hline
                P & W & Index-Mode \\ \hline
                0 & 0 & Post-index \\ \hline
                0 & 1 & Not Supported \\ \hline
                1 & 0 & Offset \\ \hline
                1 & 1 & Pre-index \\ \hline
            \end{tabular}
            
        \end{center}
    \end{frame}

    \begin{frame}{Memory Instructions}
        \begin{center}
            \renewcommand{\arraystretch}{1.2}
            % Instruction format table
            \begin{tabular}{|p{3.0em}|p{3.0em}|p{1.5em}|p{1.5em}|p{1.5em}|p{1.5em}|p{1.5em}|p{1.5em}|p{3.0em}|p{3.0em}|p{3.0em}|}
            \hline
            31:28 & 27:26 & 25 & 24 & 23 & 22 & 21 & 20 & 19:16 & 15:12 & 11:0\\
            \hline
            cond & op & \=I & P & U & B & W & L & Rn & Rd & Src2\\
            \hline
            4 bits & 2 bits & 1 bit & 1 bit & 1 bit & 1 bit &  1 bit & 1 bit & 4 bits & 4 bits & 12 bits\\
            \hline
            \end{tabular}
            \vspace{0.5cm}
        
            \begin{tabular}{|p{3.0em}|p{3.0em}|p{8.0em}|}
                \hline
                L & B & Instruction \\ \hline
                0 & 0 & STR \\ \hline
                0 & 1 & STRB \\ \hline
                1 & 0 & LDR \\ \hline
                1 & 1 & LDRB \\ \hline
            \end{tabular}
            
        \end{center}
    \end{frame}


    \begin{frame}{Example 1}
        \begin{tcolorbox}[
            enhanced,
            colback=androidBlueLight,
            colframe=androidBlue,
            arc=5pt,
            boxrule=1pt,
            title=\textbf{},
            fonttitle=\bfseries,
            coltitle=black,
            top=10pt,
            bottom=8pt,
            left=8pt,
            right=8pt,
            attach boxed title to top left={xshift=10pt, yshift=-\tcboxedtitleheight/2},
            boxed title style={
            colback=androidBlue,    
                colframe=androidBlue,
                arc=3pt,
                boxrule=0pt,
                left=6pt, right=6pt,
                top=3pt, bottom=3pt
            }
            ]
            Encode     \texttt{STR R11, [R5], \#-26}
        \end{tcolorbox}
    
    This instruction has indexing type: post-indexing: P = 0, W = 0.

    Immediate offset is subtracted from the base. So I = 0, U = 0.

    \footnotesize 
    \begin{center}
            \renewcommand{\arraystretch}{1.2}
            % Instruction format table
            \begin{tabular}{|p{3.0em}|p{3.0em}|p{1.5em}|p{1.5em}|p{1.5em}|p{1.5em}|p{1.5em}|p{1.5em}|p{3.0em}|p{3.0em}|p{7.0em}|}
            \hline
            31:28 & 27:26 & 25 & 24 & 23 & 22 & 21 & 20 & 19:16 & 15:12 & 11:0\\
            \hline
            cond & op & \=I & P & U & B & W & L & Rn & Rd & Src2\\
            \hline
            1110 & 01 & 0 & 0 & 0 & 0 &  0 & 0 & 0101 & 1011 & 0000 0001 1010\\
            \hline
            \end{tabular}
\vspace{0.5em}

            0x E405B017
    \end{center}
    
    \end{frame}
    


\begin{frame}{Example 2}
    \begin{tcolorbox}[
        enhanced,
        colback=androidBlueLight,
        colframe=androidBlue,
        arc=5pt,
        boxrule=1pt,
        title=\textbf{},
        fonttitle=\bfseries,
        coltitle=black,
        top=10pt,
        bottom=8pt,
        left=8pt,
        right=8pt,
        attach boxed title to top left={xshift=10pt, yshift=-\tcboxedtitleheight/2},
        boxed title style={
            colback=androidBlue,    
            colframe=androidBlue,
            arc=3pt,
            boxrule=0pt,
            left=6pt, right=6pt,
            top=3pt, bottom=3pt
        }
        ]
        Encode     \texttt{LDR R2, [R3, \#16]!}
    \end{tcolorbox}

    Pre-indexed Word Load:
    \begin{itemize}
        \item Load word into R2, L = 1, B = 0 (no Byte)
        \item Base register R3
        \item Immediate offset of 16, which is positive, U = 1
        \item Pre-indexed (update base register) P = 1 W = 1, P
    \end{itemize}

    \footnotesize 
    \begin{center}
        \renewcommand{\arraystretch}{1.2}
        \begin{tabular}{|p{3.0em}|p{3.0em}|p{1.5em}|p{1.5em}|p{1.5em}|p{1.5em}|p{1.5em}|p{1.5em}|p{3.0em}|p{3.0em}|p{7.0em}|}
        \hline
        31:28 & 27:26 & 25 & 24 & 23 & 22 & 21 & 20 & 19:16 & 15:12 & 11:0\\
        \hline
        cond & op & \=I & P & U & B & W & L & Rn & Rd & Src2\\
        \hline
        1110 & 01 & 0 & 1 & 1 & 0 & 1 & 1 & 0011 & 0010 & 0000 0001 0000\\
        \hline
        \end{tabular}

        \vspace{0.5em}

            0x E5B32010

    \end{center}
\end{frame}

\begin{frame}{Example 3}
    \begin{tcolorbox}[
        enhanced,
        colback=androidBlueLight,
        colframe=androidBlue,
        arc=5pt,
        boxrule=1pt,
        title=\textbf{},
        fonttitle=\bfseries,
        coltitle=black,
        top=10pt,
        bottom=8pt,
        left=8pt,
        right=8pt,
        attach boxed title to top left={xshift=10pt, yshift=-\tcboxedtitleheight/2},
        boxed title style={
            colback=androidBlue,    
            colframe=androidBlue,
            arc=3pt,
            boxrule=0pt,
            left=6pt, right=6pt,
            top=3pt, bottom=3pt
        }
        ]
        Encode     \texttt{STRB R7, [R4], R6}
    \end{tcolorbox}

    Post-indexed Byte Store:
    \begin{itemize}
        \item Store byte from R7, L = 0, Bye, hence B = 1
        \item Base register R4
        \item Offset from R6 register, no Immediate, sets U = 1
        \item Post-indexed, P = 0, W = 0
    \end{itemize}

    \footnotesize 
    \begin{center}
        \renewcommand{\arraystretch}{1.2}
        \begin{tabular}{|p{3.0em}|p{3.0em}|p{1.5em}|p{1.5em}|p{1.5em}|p{1.5em}|p{1.5em}|p{1.5em}|p{3.0em}|p{3.0em}|p{7.0em}|}
        \hline
        31:28 & 27:26 & 25 & 24 & 23 & 22 & 21 & 20 & 19:16 & 15:12 & 11:0\\
        \hline
        cond & op & \=I & P & U & B & W & L & Rn & Rd & Src2\\
        \hline
        1110 & 01 & 1 & 0 & 1 & 1 & 0 & 0 & 0100 & 0111 & 0000 0000 0110\\
        \hline
        \end{tabular}

        \vspace{0.5em}

        0x E6C47006
    \end{center}
\end{frame}


    \begin{frame}{Example 4}
        \begin{tcolorbox}[
            enhanced,
            colback=androidBlueLight,
            colframe=androidBlue,
            arc=5pt,
            boxrule=1pt,
            title=\textbf{},
            fonttitle=\bfseries,
            coltitle=black,
            top=10pt,
            bottom=8pt,
            left=8pt,
            right=8pt,
            attach boxed title to top left={xshift=10pt, yshift=-\tcboxedtitleheight/2},
            boxed title style={
                colback=androidBlue,    
                colframe=androidBlue,
                arc=3pt,
                boxrule=0pt,
                left=6pt, right=6pt,
                top=3pt, bottom=3pt
            }
            ]
            Encode     \texttt{LDR R11, [R7, R8, LSL \#2]}
        \end{tcolorbox}
    
        Word Load with Register Shift Offset:
        \begin{itemize}
            \item Load word into R11, L = 1, B = 0
            \item Base register R7
            \item Offset from R8, logically shifted left by 2, U = 1, Offset, so P = 1, W = 0
            \item Register-based offset with shift, sh = 00 for LSL
        \end{itemize}
    
        \footnotesize 
        \begin{center}
            \renewcommand{\arraystretch}{1.2}
            \begin{tabular}{|p{3.0em}|p{3.0em}|p{1.5em}|p{1.5em}|p{1.5em}|p{1.5em}|p{1.5em}|p{1.5em}|p{3.0em}|p{3.0em}|p{7.0em}|}
            \hline
            31:28 & 27:26 & 25 & 24 & 23 & 22 & 21 & 20 & 19:16 & 15:12 & 11:0\\
            \hline
            cond & op & \=I & P & U & B & W & L & Rn & Rd & Src2\\
            \hline
            1110 & 01 & 1 & 1 & 1 & 0 & 0 & 1 & 0111 & 1011 & 00010 00 0 1000 \\
            \hline
            \end{tabular}


        \vspace{0.5em}

        0x E797B108

        \end{center}
        
    \end{frame}

    \section{Branching Instruction}

\begin{frame}
    \sectionpage
\end{frame}


\begin{frame}{Encding Branching Instrunctions: B Label}

    \begin{center}
        \renewcommand{\arraystretch}{1.2}
        \begin{tabular}{|p{4.0em}|p{4.0em}|p{3.5em}|p{4.5em}|p{12.0em}|}
        \hline
        31:28 & 27:26 & 25 & 24 & 23:0\\
        \hline
        cond & op & 1 & L & imm24 \\
        \hline
        cond & 10 & 1 & L = 0 & imm24  is in 2’s complement, used to specify an instruction
        address relative to PC + 8.\\
        \hline
        \end{tabular}
    \end{center}

\end{frame}

\begin{frame}{Encding Branching Instrunctions: BL Label}

    Branch with Link

    \begin{center}
        \renewcommand{\arraystretch}{1.2}
        \begin{tabular}{|p{4.0em}|p{4.0em}|p{3.5em}|p{4.5em}|p{12.0em}|}
        \hline
        31:28 & 27:26 & 25 & 24 & 23:0\\
        \hline
        cond & op & 1 & L & imm24 \\
        \hline
        cond & 10 & 1 & L = 1 & imm24  is in 2’s complement, used to specify an instruction
        address relative to PC + 8.\\
        \hline
        \end{tabular}
    \end{center}

\end{frame}


\begin{frame}{Example 5}
    \begin{tcolorbox}[
        enhanced,
        colback=androidBlueLight,
        colframe=androidBlue,
        arc=5pt,
        boxrule=1pt,
        title=\textbf{},
        fonttitle=\bfseries,
        coltitle=black,
        top=10pt,
        bottom=8pt,
        left=8pt,
        right=8pt,
        attach boxed title to top left={xshift=10pt, yshift=-\tcboxedtitleheight/2},
        boxed title style={
            colback=androidBlue,    
            colframe=androidBlue,
            arc=3pt,
            boxrule=0pt,
            left=6pt, right=6pt,
            top=3pt, bottom=3pt
        }
        ]
        Encode     
        
        \texttt{0x80A0 BLT THERE}

        \vspace{0.5em}

        When we have the rest of the propgram as:

        \vspace{0.5em}

        \texttt{0x80A4 ADD R0, R1, R2}

        \texttt{0x80A8 SUB R0, R0, R9}

        \texttt{0x80AC ADD SP, SP, \#8}

        \texttt{0x80B0 MOV PC, LR}

        \texttt{0x80B4 THERE SUB R0, R0, \#1}

        \texttt{0x80B8 ADD R3, R3, \#0x5}

    \end{tcolorbox}

    
\end{frame}


\begin{frame}{Example 5}

    Step 1: We calculate BTA. BTA = Branch Target Address, the instruction address to be executed when the branch is
    taken. BLT has BTA of 0x80B4, the instruction address of THERE.

    \vspace{0.5em}

    Step 2: 24-bit immediate field gives the number of instructions between the BTA and PC+8 (two
    instructions past the branch)
    
    \vspace{0.5em}

    imm24 here is 3, as BTA is three instructions past PC + 8 (0x80A8).

    \vspace{0.5em}

    The processor calculates BTA from the instruction by sign-extending the 24-bit
    immediate, shifting it left 2 (to convert word to bytes) and adding it to PC-8. 

    \begin{center}
        \renewcommand{\arraystretch}{1.2}
        \begin{tabular}{|p{4.0em}|p{4.0em}|p{3.5em}|p{4.5em}|p{14.0em}|}
        \hline
        31:28 & 27:26 & 25 & 24 & 23:0\\
        \hline
        cond & op & 1 & L & imm24 \\
        \hline
        1011 & 10 & 1 & 0 & 0000 0000 0000 0000 0000 0011\\
        \hline
        \end{tabular}
    \end{center}

    
\end{frame}


\begin{frame}{Example 6}
    If the instruction at address 0x8400 is \texttt{0xEA000005}, what is the branch target address?

    \textbf{Solution:}

    \begin{itemize}
        \item This is an unconditional branch (B) instruction
        \item The immediate field is 5
    \item Branch target = PC+8 + (imm24 × 4)
    \item PC+8 for the instruction at 0x8400 is 0x8408
    \item 5 $\times$ 4 = 20 (0x14)
    \item 0x8408 + 0x14 = 0x841C
    \end{itemize}
\end{frame}

\begin{frame}{Conditions Encoding}

    \tiny

    \renewcommand{\arraystretch}{1.5}
    \begin{table}[h!]
    \centering
    \begin{tabular}{|l|l|p{6cm}|l|}
        \hline
        cond & Mnemonic & Name & CondEx \\
        \hline
        0000 & EQ & Equal & Z \\
        \hline
        0001 & NE & Not equal & $\overline{\text{Z}}$ \\
        \hline
        0010 & CS / HS & Carry set / unsigned higher or same & C \\
        \hline
        0011 & CC / LO & Carry clear / unsigned lower & $\overline{\text{C}}$ \\
        \hline
        0100 & MI & Minus / negative & N \\
        \hline
        0101 & PL & Plus / positive or zero & $\overline{\text{N}}$ \\
        \hline
        0110 & VS & Overflow / overflow set & V \\
        \hline
        0111 & VC & No overflow / overflow clear & $\overline{\text{V}}$ \\
        \hline
        1000 & HI & Unsigned higher & $\overline{\text{Z}} C$ \\
        \hline
        1001 & LS & Unsigned lower or same & Z OR $\overline{\text{C}}$ \\
        \hline
        1010 & GE & Signed greater than or equal &$\overline{N \oplus V}$ \\
        \hline
        1011 & LT & Signed less than & N$\oplus$V \\
        \hline
        1100 & GT & Signed greater than & $\overline{\text{Z}}$ $(\overline{N \oplus V})$ \\
        \hline
        1101 & LE & Signed less than or equal & Z OR (N$\oplus$$\text{V}$) \\
        \hline
        1110 & AL (or none) & Always / unconditional & Ignored \\
        \hline
    \end{tabular}
\end{table}
\end{frame}


\section{Encoding PUSH and POP Instruction}

\begin{frame}
    \sectionpage
\end{frame}

\begin{frame}{PUSH Instruction}

        \begin{tcolorbox}[
            enhanced,
            colback=androidBlueLight,
            colframe=androidBlue,
            arc=5pt,
            boxrule=1pt,
            title=\textbf{},
            fonttitle=\bfseries,
            coltitle=black,
            top=10pt,
            bottom=8pt,
            left=8pt,
            right=8pt,
            attach boxed title to top left={xshift=10pt, yshift=-\tcboxedtitleheight/2},
            boxed title style={
                colback=androidBlue,    
                colframe=androidBlue,
                arc=3pt,
                boxrule=0pt,
                left=6pt, right=6pt,
                top=3pt, bottom=3pt
            }
            ]
            
            
            \texttt{PUSH \{R4, R5, LR\} }
    
            \vspace{0.5em}
    
            is equivalent to 
    
            \vspace{0.5em}
    
            \texttt{STMFD SP!  \{R4, R5, LR\} }
    
        \end{tcolorbox}

        \footnotesize
        \renewcommand{\arraystretch}{1.2}
        \begin{tabular}{|p{3.0em}|p{3.0em}|p{1.5em}|p{1.5em}|p{1.5em}|p{1.5em}|p{1.5em}|p{1.5em}|p{3.0em}|p{10.0em}|}
        \hline
        31:28 & 27:26 & 25 & 24 & 23 & 22 & 21 & 20 & 19:16 & 15:0\\
        \hline
        cond & op & I & P & U & S & W & L & Rn & register list\\
        \hline
        1110 & 10 & 0 & 1 & 0 & 0 & 1 & 0 & 1101 & 0100000000110000 \\
        \hline
        \end{tabular}

        \vspace{0.5em}
    

        P = 1 Pre-indexing mode, U = 0 for decrement, W = 1 write back, updates SP after storing, L = 0 for STM (Store operation)

        Rn is the base register which is SP. Register list uses bit masking. Hence, encoding is 0x E92D4030.

        
\end{frame}

\begin{frame}{POP Instruction}

    \begin{tcolorbox}[
        enhanced,
        colback=androidBlueLight,
        colframe=androidBlue,
        arc=5pt,
        boxrule=1pt,
        title=\textbf{},
        fonttitle=\bfseries,
        coltitle=black,
        top=10pt,
        bottom=8pt,
        left=8pt,
        right=8pt,
        attach boxed title to top left={xshift=10pt, yshift=-\tcboxedtitleheight/2},
        boxed title style={
            colback=androidBlue,    
            colframe=androidBlue,
            arc=3pt,
            boxrule=0pt,
            left=6pt, right=6pt,
            top=3pt, bottom=3pt
        }
        ]
        
        
        \texttt{POP \{R4, R5, LR\} }

        \vspace{0.5em}

        is equivalent to 

        \vspace{0.5em}

        \texttt{LDMFD SP!  \{R4, R5, LR\} }

    \end{tcolorbox}

    \footnotesize
    \renewcommand{\arraystretch}{1.2}
    \begin{tabular}{|p{3.0em}|p{3.0em}|p{1.5em}|p{1.5em}|p{1.5em}|p{1.5em}|p{1.5em}|p{1.5em}|p{3.0em}|p{10.0em}|}
    \hline
    31:28 & 27:26 & 25 & 24 & 23 & 22 & 21 & 20 & 19:16 & 15:0\\
    \hline
    cond & op & I & P & U & S & W & L & Rn & register list\\
    \hline
    1110 & 10 & 0 & 0 & 1 & 0 & 1 & 1 & 1101 & 0100000000110000 \\
    \hline
    \end{tabular}

    \vspace{0.5em}


    P = 0 Post-indexing mode, U = 1 for increment, W = 1 write back, updates SP after loading, L = 1 for LDM (Load operation)

    Rn is the base register which is SP. Register list uses bit masking. Hence, encoding is 0x E8BD4030.

    
\end{frame}





    \begin{frame}
        \Huge{\centerline{\color{androidGreen}\textbf{The End}}}
    \end{frame}
    
    
\end{document}